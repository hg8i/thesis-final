\chapter{Search for Non-resonant Signatures and Contact Interactions}\label{sec:ci}

% Physical processes that produce a pair of final state leptons offer a clear window into the dynamics of high energy collisions.
Observations of pairs of leptons, both of electron and muon flavors, offer a clear window into the dynamics of high energy collisions.
The clarity of this window is due to the long lifetime and ease of detection offered by those leptons.
Of particular use is the invariant-mass spectra of dilepton pairs, which elucidates the possible mechanisms of their production by means of local enhancements, or resonances. 
This has proved a useful tool that has been exploited throughout the history of experimental particle physics.
In 1974, a group working at Brookhaven National Laboratory\cite{jpsiBnl} and another group working at the Stanford Linear Accelerator Center\cite{jpsiSlac} used the dielectron mass spectrum to independently discover the \jpsi resonance at \mee=3.1~GeV.
In 1977, a group working at Fermilab used the dimuon mass spectrum to discover the $\Upsilon$ resonance at \muu=9.5~GeV \cite{upsilon}.
In 1983, the UA1 group working at the SPS collider at CERN used both dielectron and dimuon events to detect the decay of the \Z boson at a mass of $\mll\approx95$~GeV\cite{z0ua1}.
Later in the same year, the UA2 group used dielectron events to produce a measurement of \mee=91.9~GeV.
The utility of the dilepton final state is derived from the fact that the final state consisting of two leptons is fully reconstructible.

\begin{figure}[htp]
\captionsetup[subfigure]{position=b}
\centering
\includegraphics[width=0.7\textwidth]{figures/ci/lowmass/lowmass-lower.pdf}
\caption{The invariant-mass spectra around the \Z boson mass peak from one million collisions with two muons.}
\label{fig:ciLowMass}
\end{figure}

% General non-resonant
The discoveries made with the invariant-mass spectra associate enhancement in rate of observed events with new mechanisms responsible for the enhancement.
These enhancements may be localized, as is the case for narrow resonances produced by \jpsi decay.
Alternatively, broad enhancements are possible as well; such enhancements, termed \emph{non-resonant}, are the focus of this search.
\footnote{Broad deficits are possible as well; one example is the observations of negative (positive) interference between $\gamma^*$ and \Z that changes the forward-backward charge symmetry below (above) the \Z mass peak in electron-positron experiments \cite{fbasym}.}
The investigation presented in this chapter contemplates the dielectron and dimuon invariant mass spectra in search for new and broad excesses appearing in the high mass tail.

Many new physics models beyond the Standard Model predict broad enhancements in dilepton production.
% Contact interactions
A particularly interesting cause of a non-resonant signature is a \emph{contact interaction} (CI) between quarks and leptons at an energy scale exceeding that of the collision energy.
Although direct resonant production is inaccessible, a new contact interaction would lead to off-shell production and interference with the SM production.
This can be caused by a mediator particle with a mass far above the \sqrts=13~TeV collision energies offered by the LHC.
A \qqll CI is also interesting because it is a necessary outcome of quarks and leptons sharing a common substructure \cite{eichten}.

Many new physics models outside the Standard Model predict non-resonant excesses.
To maintain a degree of generality and model independence, the products of this search are designed to be agnostic as to the underlying mechanism behind the feature.
As a result, when possible, the procedure used to produce results tends to limit or exclude the role played by signal models.
Due to the nature of the chosen analysis strategy, several choices must be made with respect to the region of date in which the search is conducted.
In these cases, the analysis is designed in order to optimize sensitivity to a generic formulation of contact interaction that serves as a benchmark.
This model dependant optimization remains, for the greater part, separate from the final results.

\begin{figure}[h!]
\captionsetup[subfigure]{position=b}
\centering
\includegraphics[width=0.99\textwidth]{figures/ci/massRanges.pdf}
\caption{
A schematic example of the dilepton invariant mass spectrum. 
The monotonically falling total background shape is shown by the solid black line, while the dotted red line shows an example of a CI signal plus the total background shape.
A background model is fit to the data it in a low-mass control region (shaded blue area) where a potential bias from the presence of a signal is negligible.
The resulting background shape is extrapolated from the control region into the high-mass signal region (shaded red area).
% The gap illustrated between the CR and the SR is found to be the preferred case for the destructive interference cases only.
}
\label{fig:ciStrat}
\end{figure}

% Challenges
This search for contact interactions is carried out using the dielectron and dimuon invariant-mass spectra.
Since CI produce final states of the same topology as some SM backgrounds, the signal production interferes either constructively or destructively with these backgrounds.
In the constructive case, the CI strictly enhances the spectra with a broad non-resonant shape.
In the destructive case, the interference modifies the background spectra as illustrated in Figure \ref{fig:ciStrat}. 
Both types of interference are considered in this analysis.

This search is complicated by both experimental and phenomenological challenges.
It involves probing the highest energy regimes and smallest length scales ever accessed by observation.
The first challenge results from the width of the non-resonant shape of interest, as shown in the red line of Figure \ref{fig:ciStrat}.
This shape is qualitatively similar to the shape of the background, so care must be taken to avoid interpreting an existing signal as background, thereby losing sensitivity to new physics.
The second challenge is the focus on new CI signals that may manifest themselves in the tail of the invariant-mass spectra.
Attention must be paid to systematic uncertainties in this regime, as well as to the relevant resolution of the measured spectra.
The third challenge is in modeling statistical knowledge of the background in signal regions with very low occupancy.
The sensitivity of the analysis is similarly impacted by statistical and experimental uncertainties on the background expectation, as these are of similar magnitude of the background itself.
New techniques are required to properly address these challenges.

% new analysis strategy: data-driven
This analysis introduces a number of changes that depart from previous searches.
The result presented here is the first non-resonant dilepton search at the LHC to use a functional form fit to data to estimate the background, rather than a background estimate derived from simulated events.
This choice removes the dependence of the background on the theoretical assumptions involved in the simulation process.
This is important because these assumptions are both significant and poorly constrained in the high-mass regime.

% Solution: avoid fitting signal shape
The contact interactions that provide a benchmark model predict deviations from the expected gradient of the high-mass tail of the dilepton mass spectrum; the subtlety of this effect could easily be masked by a background description of sufficient flexibility to match the data.
Therefore, the background event distribution at high masses is estimated based on a low-mass control region (CR) where the contribution of the benchmark signals are negligible.
A functional form is fit to the observed data in the CR, and extrapolated to higher masses to model the production rate of background events.
The search is then performed in a high-mass signal region (SR); here, event production by the benchmark signals is predicted to dominate over the background production.
The arrangement of these mass ranges is illustrated in Figure \ref{fig:ciStrat}.
The extrapolation from CR to SR avoids fitting the data in the regime where CI signals could potentially perturb the fit.
This strategy reduces the impact of a signal shape, if present in the data, on the fitted background model.

% Solution: avoid resolution, etc
ATLAS measures electron transverse energy, \et, and muon transverse momentum, \pt.
Consequently the invariant-mass resolution of dielectron pairs depends on \et resolution.
For high-\et electrons, this grows as a constant fraction of \et. 
Meanwhile, the invariant-mass resolution of dimuon pairs depends on \pt.
As a result, the fractional muon resolution grows linearly with \pt.
Both of these resolutions propagate to their respective invariant-mass spectra.
A single-bin SR is used to combat the effects of these resolutions, which are particularly impactful in the high-mass regime.
All events falling within the SR are counted identically, which mitigates the importance of \et and \pt resolution.
This approach has the additional benefit of removing shape information in the signal region, making the results readily reinterpretable for signal models predicting different non-resonant shapes.

% new analysis strategy: Bayesian
A key part of this analysis is the statistical treatment of the observations.
The Bayesian statistics used in previous ATLAS and CMS searches for CI has been replaced frequentist statistical framework.
This removes the dependence of the result on prior probabilities to observe a signal.
If the interference between signal and SM processes is not negligible, as is the case for CI, the choice of one prior over another is poorly motivated~\cite{Aad:2012hf,EXOT-2016-05}.
% The choice to analyse the observations using frequentist statistics produces in more robust results.

Finally, the transition to a background estimation from the data exchanges the systematic uncertainties in theoretical predictions for statistical uncertainties in data.
There are three new uncertainties that arise from this approach.
The dominant uncertainty in the expected background is due to statistical fluctuations in the CR.
Next in importance is the uncertainty in the degree to which the extrapolation from the CR can produce a background estimate different from the underlying distribution. Such a difference leads to a signal-like deflection in the SR. This uncertainty is quantified using the simulated background and the associated uncertainties.
The third and smallest uncertainty describes the impact of potential signal contamination in the CR.

Two signal regions are considered for each lepton flavor, leading to four signal regions in total.
For each flavor, the first SR extends to relatively lower invariant-mass and targets CI that interfere constructively with the SM.
The second SR remains at relatively high invariant-mass and targets CI that interfere destructively with the SM.
For the latter case, a gap is left between the CR and SR in order to avoid counting the destructive interference in the SR, as illustrated in Figure \ref{fig:ciStrat}.

A statistical analysis is performed on the observation in each SR.
The first results of the analysis are limits on the \xsbr in each SR, which can readily be reinterpreted in terms of various new physics models, without limitation to contact interactions.
This result is the first of its kind, and is a new development for non-resonant searches at the LHC.
The second result uses these four signal regions to set limits on CI models.
These are produced to be reinterpretable in terms of arbitrary CI models \cite{chala}.
The results of this analysis were published on November 4, 2020 \cite{ciAaron}.

% Chapter outline
This chapter describes the ATLAS search for contact interactions using the Run~2 dataset.
Section \ref{sec:ciMotivation} discusses the theoretical motivation.
Section \ref{sec:ciEvSel} describes the selection of data used for the search.
Sections \ref{sec:ciSig} and \ref{sec:ciBkg} present the signal and background models, respectively.
Next Section \ref{sec:ciSyst} discusses the systematic uncertainties used in the result.
Section \ref{sec:ciStat} details the statistical analysis of the data.
Finally Section \ref{sec:ciResults} presents the results and Section \ref{sec:ciConclusion} summarizes the analysis.

\section{Theoretical Motivation for Non-resonance Signatures}\label{sec:ciMotivation}

The effects of a new interaction may be observed at an energy much lower than that required to produce direct evidence of the existence of a new gauge boson.
The charged weak interaction responsible for nuclear $\beta$ decay provides such an example.
A non-renormalizable description of this process was formulated by Fermi in the form of a four-fermion contact interaction~\cite{Fermi:1934hr}.
A CI can also describe deviations from the SM in proton--proton scattering due to quark and lepton compositeness, where a characteristic energy scale \lam corresponds to the binding energy between fermion constituents.
The following Lagrangian can describe a generic \llqq contact interaction, including fermion compositeness \cite{eichten, Eichten:1984eu};
% A new interaction or compositeness in the process $q\overline{q} \to \ell^+\ell^-$ can be described by the following four-fermion contact interaction Lagrangian~\cite{eichten, Eichten:1984eu},

\begin{equation}\label{eqn:ciLagrangian}
\begin{array}{r@{\,}c@{}c@{\,}l@{\,}l}
\mathcal L = \frac{g^2}{2\Lambda^2}\;[ && \eta_{\mathrm{LL}}&\, (\overline q_{\mathrm L}\gamma_{\mu} q_{\mathrm L})\,(\overline\ell_{\mathrm L}\gamma^{\mu}\ell_{\mathrm L}) \nonumber \\
& +&\eta_{\mathrm{RR}}& (\overline q_{\mathrm R}\gamma_{\mu} q_{\mathrm R}) \,(\overline\ell_{\mathrm R}\gamma^{\mu}\ell_{\mathrm R}) \\
&+&\eta_{\mathrm{LR}}& (\overline q_{\mathrm L}\gamma_{\mu} q_{\mathrm L}) \,(\overline\ell_{\mathrm R}\gamma^{\mu}\ell_{\mathrm R}) \\
&+&\eta_{\mathrm{RL}}& (\overline q_{\mathrm R}\gamma_{\mu} q_{\mathrm R}) \,(\overline\ell_{\mathrm L}\gamma^{\mu}\ell_{\mathrm L})& ] \: ,\nonumber
\end{array}
\end{equation}

\noindent where $g$ is a coupling constant chosen by convention\footnote{The interested reader may note that this convention, followed in all ATLAS results, may be adjusted by consistent multiplication of \lam.} to satisfy $g^2/4\pi = 1$, \lam is the contact interaction scale, and $q_{\mathrm L,R}$ and $\ell_{\mathrm L,R}$ are left-handed and right-handed quark and lepton fields, respectively.
The parameters $\eta_{ij}$, where $i$ and $j$ may be left (L) or right (R), define the chiral structure of the new interaction.
Different chiral structures are considered by choices of the coefficients $\eta_{ij}$. For example, the left-right model is obtained by setting $\eta_{\mathrm{LR}} = \pm 1$ and $\eta_{\mathrm{RL}} = \eta_{\mathrm{LL}} = \eta_{\mathrm{RR}} = 0$.
Likewise, the left-left (LL), right-left (RL), and right-right (RR) chirality models are correspond to Lagrangians with the corresponding $\eta_{ij}$ set to $\pm 1$, and the remaining $\eta_{ij}=0$.
The sign of $\eta_{ij}$ determines the sign of interference: $\eta_{ij}=-1$ results in constructive interference, while  $\eta_{ij} = +1$ results in destructive with the Standard Model. 

Equation \ref{eqn:ciLagrangian} becomes more specific in the context of \llqq CI searches in dilepton final states at the LHC.
The terms take the form $\eta_{ij}\left(\bar{q}_i\gamma_{\mu}q_i\right)\left(\bar{\ell}_j\gamma^{\mu}\ell_j\right)$, where $q_i$ and $\ell_j$ are the quark and lepton fields respectively.
The differential cross-section for the process $q\bar{q}\rightarrow\ell^+\ell^-$, in the presence of CI, can be separated into the SM DY term plus terms involving the new contact interaction.
% This separation is given in Equation \ref{eqn:cixs}.
\begin{equation}
\frac{\text{d}\sigma}{\text{d}m_{\ell\ell}} = \frac{\text{d}\sigma_\textrm{DY}}{\text{d}m_{\ell\ell}} - \eta_{ij}\frac{F_\textrm{I}}{\Lambda^2} + \frac{F_\textrm{C}}{\Lambda^4}
\label{eqn:cixs}
\end{equation}
In Equation \ref{eqn:cixs}, the first term accounts for the DY process, the second term corresponds to the interference between the DY and CI processes, and the third term corresponds to the pure CI contribution.
The latter two terms include $F_\textrm{I}$ and $F_\textrm{C}$, respectively, which are functions of the differential cross-section with respect to $m_{\ell\ell}$ with no dependence on \lam~\cite{Eichten:1984eu}.
The interference may be constructive or destructive, and it is determined by the sign of $\eta_{ij}$.

\begin{figure}[htb]
\begin{center}
\begin{equation}
\begin{tikzpicture}
% \draw[style=help lines] (-3,-5) grid (9,2);
\draw (-2,-2) -- (-2,2);
\draw (9.3,-2) -- (9.3,2);
\node (A) at (9.5,1.9) {2};
\begin{feynman}[medium]
    \vertex (p1);
    \vertex [right=3.6em of p1] (p2);
    \vertex [above left=of p1] (qa){\(\phantom{^+}\qbar\)};
    \vertex [below left=of p1] (qb){\(\phantom{^+}q\)};
    \vertex [above right=of p2] (h){\(\ell^{-}\phantom{\qbar}\)};
    \vertex [below right=of p2] (v){\(\ell^{+}\phantom{\qbar}\)};
    \diagram* {
    (qb) --[fermion] (p1) --[fermion] (qa),
    (p1) --[boson,edge label=\(\gamma^*/Z\)] (p2),
    (v) --[fermion] (p2) --[fermion] (h),
    };
\end{feynman}
\node (A) at (10.5em,0) {+};
\begin{feynman}[xshift=17em,medium]
    \vertex [blob] (p1){\(\lam\)};
    \vertex [above left =4.80em of p1] (qa){\(\phantom{^+}\qbar\)};
    \vertex [below left =4.80em of p1] (qb){\(\phantom{^+}q\)};
    \vertex [above right=4.80em of p1] (h){\(\ell^{-}\phantom{\qbar}\)};
    \vertex [below right=4.80em of p1] (v){\(\ell^{+}\phantom{\qbar}\)};
    \diagram* {
    (qb) --[fermion] (p1) --[fermion] (qa),
    (v) --[fermion] (p1) --[fermion] (h),
    };
\end{feynman}
\end{tikzpicture}
\end{equation}
\end{center}
\vspace{-.4cm}
\caption{Leading-order production mechanism for Drell-Yan with additional contact term with scale \lam in the dilepton final state.}
\label{FeynmanCI}
\end{figure}

Contact interactions have motivated a rich set of searches.
Numerous searches for CI have been carried out in neutrino--nucleus and electron--electron scattering~\cite{Anthony:2005pm}, as well as electron--positron~\cite{Abdallah:2008ab, Schael:2006wu}, electron--proton~\cite{Aaron:2011mv}, and proton--antiproton colliders~\cite{Abulencia:2006iv,Abazov:2009ac}.
Searches for CI have also been performed by the ATLAS and CMS Collaborations~\cite{Aad:2014wca, Khachatryan:2014fba}.
The strongest exclusion limits for \llqq CI come from the previous ATLAS non-resonant dilepton analysis conducted using 36.1\fb of \sqrts=13~TeV proton--proton data \cite{Aaboud:2016cth}.
Other ATLAS studies of note include the 2012/2014 search for contact interactions using \sqrts=7/8~TeV collisions at ATLAS \cite{EXOT-2013-19, EXOT-2012-17}.

\section{Dilepton Event Selection}\label{sec:ciEvSel}
The present search is concerned with collisions that produce pairs leptons.
This section lists selection criteria used to identify such events.
% from the dataset collected during Run~2 of the LHC.
The observed dataset, which consists of the events collected by ATLAS during  Run~2 of the LHC, is detailed along with the corresponding simulated background and signal datasets.
% Finally, comparisons between the recorded data and simulation are provided.


\subsection{Event Selection}
During Run~2, roughly $10^{16}$ proton collisions took place inside the ATLAS experiment.
The majority of these events are uninteresting for the purpose of this analysis, so only events meeting appropriate criteria are considered.
This reduces the total number of data events considered for the analysis to 754,870 dimuon events and 883,594 dielectron events.

% GRL
Only events recorded during good operation of the detector are used.
The events meeting this requirement comprise the Good Run List, summarized in Section \ref{sec:physData}.

% Trigger
The first requirement for an event to be considered is the trigger: only events identified as interesting by the ATLAS trigger system are recorded.
The triggers used during data collection differ from year to year. 
In the dielectron channel, the following trigger requirements are applied.
\begin{itemize}
	\item 2015: Two electrons with $\et>12$~GeV,
	\item 2016: Two electrons with $\et>17$~GeV,
	\item 2017 and 2018: Two electrons with $\et>24$~GeV.
	% \item 2015: 2e12\_lhloose\_L12EM10VH
	% \item 2016: 2e17\_lhvloose\_nod0
	% \item 2017 and 2018: 2e24\_lhvloose\_nod0
\end{itemize}
Although events passing these triggers are expected to contain two electrons, both may not be reconstructed after the event is fully processed. 
Therefore, subsequent criteria require at least two electrons to be reconstructed.

In the dimuon channel, the following trigger requirements are applied.
\begin{itemize}
	\item 2015: One isolated ($\ptconeThirty/\pt<0.06$) muon with $\pt>26$~GeV, or any non-isolated muon with $\pt>50$~GeV,
	\item 2016, 2017 and 2018: The same requirement, except the isolation uses \ptvarconeMuon.
	% \item 2015: mu26\_imedium or mu50
	% \item 2016, 2017 and 2018: mu26\_ivarmedium or mu50
\end{itemize}
These trigger on events with single isolated muons.
These triggers are used, rather than a muon equivalent to the electron triggers, to increase the trigger's efficiency for dimuon events; the requirement for an event to have two muons is enforced in the later.

% Object selection
After passing the trigger requirement, events are evaluated under selection criteria.
In events where multiple vertices are reconstructed, the vertex with the largest $\sum\pt^2$ defines the \emph{primary vertex}.
Events are required to have at least two Inner Detector tracks associated with the primary vertex.
The first step is to define requirements for which physical objects are to be considered in each event. This step follows the object definitions from Section \ref{sec:physObjects}.
% Many of the terms used here follow the definitions found in that section.

Further requirements are made as to where the objects were reconstructed in the detector. 
This defines the fiducial region in which the search is carried out.
This definition differs for electrons and muons.

Electrons are defined using the \code{Medium} likelihood identification.
They are required to pass \code{Gradient} isolation.
Additionally, they must not be from a dead calorimeter cluster.
An additional \emph{loose selection} for electrons is defined to study the background from objects falsely reconstructed as electrons.
For these electrons, the \code{LooseAndBLayer} LH identification replaces the \code{Medium} LH.
This is otherwise the same as the nominal electron selection.
The kinematic criteria for both electron selections are listed in Table \ref{tab:ciElectronSel}.

\begin{table}[!htb]
\caption{Selection criteria for electrons. Parameters $d_{0}$ and $z_{0}$ are the transverse and longitudinal displacements of the track associated with the electron, and the vertex.}
\begin{center}
    \begin{tabular}[ht]{l l}
    \toprule
    Feature & Criteria \\
    \midrule
    Pseudorapidity range & $(|\eta| < 1.37) \quad || \quad (1.52 < |\eta| < 2.47)$ \\
    Transverse momentum & p$_T$ $>$ 30~GeV \\
    Track impact parameter significance & ${|d_{0}^{BL}|\over\sigma}$ $<$ 5 \\
    Track $z$ displacement & $|\Delta z_{0}^{BL} \sin{\theta}| <$ 0.5~mm \\
    \bottomrule
    \end{tabular}
\end{center}
\label{tab:ciElectronSel}
\end{table}

Muons are defined using the \code{High}-$p_T$ selection working point and must pass the isolation requirement \code{FCTightTrackOnly}.
An additional cut, the bad muon veto, is used to reject muons with poorly measured \pt.
The remaining kinematic criteria for muons are given in Table \ref{tab:ciMuonsSel}.

\begin{table}[ht]
\caption{Selection criteria for muons. Parameters $d_{0}$ and $z_{0}$ are the transverse and longitudinal displacements of the track associated with the muon, and the vertex.}
\begin{center}
    \begin{tabular}[ht]{l l}
    \toprule
    Feature & Criteria \\
    \midrule
    Transverse momentum  & $\pt>30$ GeV\\
    Pseudorapidity range & $|\eta|<2.5$ \\
    Track impact parameter significance & ${|d_{0}^{BL}|\over\sigma}< 3$ \\
    Track $z$ displacement  & $|\Delta z_{0}^{BL} \sin{\theta}| < 0.5~mm$\\
    \bottomrule
    \end{tabular}
\end{center}
\label{tab:ciMuonsSel}
\end{table}

Occasionally, the interaction of a single particle with detectors results in the reconstruction of two particles.
To limit this occurrence, an \emph{overlap removal} scheme removes particles that are suspiciously close to each other.
The criteria are listed in Table \ref{tab:ciOr}.
\begin{table}[ht]
\caption{Overlap removal}
\begin{center}
    \begin{tabular}[ht]{l l l}
    \toprule
    Reject & Against & Criteria \\
    \midrule
    Electron & Electron & Shared ID track, $\pt^1<\pt^2$ \\
    Muon     & Electron & Is calo-muon and shared ID track \\
    Electron & Muon     & Shared ID track \\
    \bottomrule
    \end{tabular}
\end{center}
\label{tab:ciOr}
\end{table}
Further rejection of muons and electrons takes place if a jet is reconstructed within an angular distance $\Delta R<0.4$.
This helps reduce the presence of secondary leptons.

% Event selection
These criteria reduce the full set of recorded events to a subset to consider, and within each event a set of physical objects to analyze.
It remains to determine whether the event is interesting for the purpose of this dilepton analysis.
Only events containing either two electrons or two muons meet this threshold.
Of the same-flavor leptons in the event, the leading and subleading \et (\pt) pair are selected in the electron (muon) channel.
In the muon channel, only pairs of oppositely charged muons are considered. 
In the electron channel, the charge is ignored because bremsstrahlung emission of photons.
Such photons can alter the track of an electron, leading to the mis-identification of its charge.
Finally, a preliminary invariant mass cut of $\mll>130$~GeV is required.
In the case where both a dielectron and dimuon candidate meet these requirements, the dielectron is selected, and the dimuon is discarded.
This choice is made due to the superior resolution for high-\et electrons.

\subsection{Data and Simulation}
% The data yield rate, broken into the different runs and periods for each year, are shown in Figure ~\ref{fig:ciYields}.
% These plots count events after applying the full selections.

% \begin{figure}[ht!]
% \captionsetup[subfigure]{position=b}
% \centering
% \subfloat[][]{{\includegraphics[width=0.48\textwidth]{figures/ci/dataMc/compare_data_yields2015.pdf}}}
% \subfloat[][]{{\includegraphics[width=0.48\textwidth]{figures/ci/dataMc/compare_data_yields2016.pdf}}}\\
% \subfloat[][]{{\includegraphics[width=0.48\textwidth]{figures/ci/dataMc/compare_data_yields2017.pdf}}}
% \subfloat[][]{{\includegraphics[width=0.48\textwidth]{figures/ci/dataMc/compare_data_yields2018.pdf}}}
% \caption{Data yields for the each run period for the inclusive $ee$ (above) and $\mu\mu$ (below) selections.}
% \label{fig:ciYields}
% \end{figure}
% \clearpage

The data used in this analysis were collected during the LHC Run 2 from \sqrts=13~TeV proton-proton collisions.
The recorded integrated luminosity of the collisions is $139.0\pm2.4$~\fb \cite{ATLAS-CONF-2019-021}.

Despite the reliance on background estimates derived from data, this analysis uses simulated invariant-mass distributions for three purposes.
The first use is to model the CI signal. This is done using simulated DY events, reweighted to include interference and direct production from a contact interaction.
The second use is to test a variety of choices made during the analysis. In particular, the simulation informs the choice of a functional form that matches the expected background shape. Simulation is also used to optimally select the control and signal regions to maximize expected sensitivity while avoiding potential biases.
The third use is to measure the impact of experimental and theoretical uncertainties on the results.

All simulation-based background contributions are scaled by their respective cross-sections and summed to obtain the simulated background invariant-mass distribution.
The main backgrounds in decreasing order of contribution to the full mass spectrum are the Drell--Yan (DY) process, top-quark pair production ($t\bar{t}$), single-top-quark production, and diboson production.
The multi-jet and $W+$jets processes in the dielectron channel are estimated from the data using the matrix method \cite{EXOT-2016-05}. The contribution of such processes to the analysis is estimated using a likelihood fit, and is later treated as an uncertainty in the simulated background.
The same processes in the dimuon channel, as well as processes with $\tau$-leptons in both channels, have been measured to have a negligible impact and consequently are not considered.
The event generators for the hard-scattering process and the programs used for parton showering are listed in Table~\ref{tab:MC} with their respective parton distribution functions (PDFs).
Afterburner generators such as \textsc{Photos}~\cite{Golonka:2005pn} for the final-state photon radiation (FSR) modeling, \textsc{MadSpin}~\cite{Artoisenet:2012st} to preserve top-quark spin correlations, and \textsc{EvtGen}~\cite{Lange:2001uf} for the modeling of $c$- and $b$-hadron decays, are also included in the simulation.

\begin{table}[htbp]
\caption{The programs and PDFs used to generate the hard-scatter matrix element (ME) and to simulate parton showering (PS) in the signal and background processes.
\centering
The top-quark mass is set to 172.5 GeV.}
{\scriptsize
\begin{tabular}{lll}
\toprule
Background Process & ME Generator and ME PDFs & PS and non-perturbative effect with PDFs \\\hline
NLO Drell--Yan & \POWHEGBOX~, CT10~, \textsc{Photos} & \PYTHIAV{v8.186}~, CTEQ6L1~, \textsc{EvtGen1.2.0} \\
$t\bar{t}$  & \POWHEGBOX, NNPDF3.0NLO~ & \PYTHIAV{v8.230}, NNPDF23LO~, \textsc{EvtGen1.6.0} \\
Single top $s$-channel, $Wt$& \POWHEGBOX, NNPDF3.0NLO & \PYTHIAV{v8.230}, NNPDF23LO, \textsc{EvtGen1.6.0} \\
Single top $t$-channel & \POWHEGBOX, NNPDF3.04fNLO, \textsc{MadSpin} & \PYTHIAV{v8.230}, NNPDF23LO, \textsc{EvtGen1.6.0}  \\
Diboson ($WW$, $WZ$ and $ZZ$) & \SHERPA 2.1.1~, CT10 &\SHERPA 2.1.1, CT10  \\\hline
Signal Process & & \\\hline
LO Drell--Yan & \PYTHIAV{v8.186}, NNPDF23LO  &  \PYTHIAV{v8.186}, NNPDF23LO, \textsc{EvtGen1.2.0} \\
LO CI & \PYTHIAV{v8.186}, NNPDF23LO  &  \PYTHIAV{v8.186}, NNPDF23LO, \textsc{EvtGen1.2.0} \\
\bottomrule
\end{tabular}
}
\normalsize
\label{tab:MC}
\end{table}


The DY~\cite{ATL-PHYS-PUB-2016-003} and diboson~\cite{ATL-PHYS-PUB-2016-002} samples were generated in slices of dilepton mass to increase the sample statistics in the high-mass region.
Next-to-next-to-leading-order (NNLO) corrections in QCD and next-to-leading-order (NLO) corrections in EW were calculated and applied to the DY events.
The corrections were computed with {\textsc{VRAP}} v0.9~\cite{vrap} and the CT14 NNLO PDF set~\cite{CT14} in the case of QCD effects, whereas they were computed with {\textsc{MCSANC}}~\cite{MCSANC} in the case of quantum electrodynamic effects due to initial-state radiation, interference between initial- and final-state radiation, and Sudakov logarithm single-loop corrections.
These are calculated as mass-dependent scale factors that are applied to simulated events before reconstruction.
The top-quark samples~\cite{ATL-PHYS-PUB-2016-020} are normalized to the cross-sections calculated at NNLO in QCD, including resummation of the next-to-next-to-leading logarithmic soft gluon terms using \textsc{Top++}2.0 \cite{Czakon:2011xx}.

All fully simulated event samples include the effect of multiple proton interactions in the same or neighboring bunch crossings.
These effects are collectively referred to as pile-up.
The simulation of pile-up collisions was performed with \PYTHIAV{v8.186} using the ATLAS A3 set of tuned parameters~\cite{ATL-PHYS-PUB-2016-017} and the NNPDF23LO PDF set, and weighted to reproduce the average number of pile-up interactions per bunch crossing observed in data.
The generated events were passed through a full detector simulation~\cite{SOFT-2010-01} based on\ \GEANT~4~\cite{geant}.

% Scale factors
The simulated data is weighted by several scale factors (SF) to improve its representation of the observed data.
Pile-up weights are used to describe the effects multiple collisions per beam crossing.
Mass dependant $K$-factors account for differences in the total cross-section if higher-order calculations are available for a given process compared to the order available for simulation. In the case of the LO and NLO DY samples, the SFs provide corrections for higher-order QCD, EW, and photon-induced (PI) effects.
Experimental scale factors for leptons are considered. For electrons, reconstruction, trigger, isolation, and identification scales factors are applied. For muons, reconstruction, trigger, isolation, and track-to-vertex association (TTVA) scale factors are applied.
Trigger scale factors according to the specific channel.

% \begin{itemize}
%     \item Pile-up weights describe the effects multiple collisions per beam crossing.
%     \item Mass dependant $K$-factors account for differences in the total cross-section if higher-order calculations are available for a given process compared to the order available for simulation. In the case of the LO and NLO DY samples, the SFs provide corrections for higher-order QCD, EW, and photon-induced (PI) effects.
%     \item Experimental scale factors for leptons are considered. For electrons, reconstruction, trigger, isolation, and identification scales factors are applied. For muons, reconstruction, trigger, isolation, and track-to-vertex association (TTVA) scale factors are applied.
%     \item Trigger scale factors according to the specific channel.
% \end{itemize}

To reduce statistical uncertainties, a large additional DY sample is used where the detector response is modeled by smoothing the dilepton invariant-mass with mass-dependent acceptance and efficiency corrections, instead of using the computationally expensive \GEANT~4 simulation.
The relative dilepton mass resolution used in the smearing procedure is defined as $(m_{\ell\ell}-m_{\ell\ell}^\mathrm{true})/m_{\ell\ell}^\mathrm{true}$, where $m_{\ell\ell}^\mathrm{true}$ is the generated dilepton mass at Born level before final-state radiation.
The mass resolution is parameterized as a sum of a Gaussian distribution, which describes the detector response, and a Crystal Ball function composed of a secondary Gaussian distribution with a power-law low-mass tail, which accounts for bremsstrahlung effects and for the effect of poor resolution in the muon momentum at high-\pt.
The parameterization of the relative dilepton mass resolution as a function of $m_{\ell\ell}^\mathrm{true}$ is determined by a fit of the function described above to simulated DY events at NLO.
A similar procedure is used to produce a mass-smeared $t\bar{t}$ sample.
These two samples replace the equivalent ones produced with the full detector simulation wherever applicable in the remainder of the analysis.
These samples are composed of over 55 times the number of events in the observed dataset.

Signal \mll distribution shapes are obtained by a matrix element reweighting of the leading-order (LO) DY samples generated in slices of dilepton mass \cite{EXOT-2016-05}.
This reweighting includes the full interference between the non-resonant signal and the background DY process.
The weight function is the ratio of the analytical matrix elements of the full contact interaction (including the DY component) and the DY process only, both at LO precision.
It takes as an input the generated dilepton mass at Born level before FSR, the flavor of the incoming quarks, and the CI model parameters (\lam, chirality states, and the sign of interference).
These weights are applied to the LO DY events to transform these into the CI signal shapes, in steps of $2$~TeV between $\Lambda=12$~TeV and $\Lambda=100$~TeV.
Mass-dependent higher-order QCD production corrections for the signals are computed with the same methodology as for the DY background, correcting from LO to NNLO precision.
Similarly, electroweak corrections for the signals are applied in the CI reweighting along with the interference effects, correcting from LO to NLO precision.
These signal shapes are used for optimizations as well as for calculations of the cross-section and acceptance times efficiency.

The invariant-mass distributions of the simulated datasets, and of the observed data are shown in Figure \ref{fig:ciMassMcPlot}.
Several representative contact interaction shapes imposed on top of the background yields for reference.
These plots clearly show the relative composition of the background in the simulated distributions.
The plots of the \ee selection additionally include the multi-jet and $W+$jets background.

\afterpage{
\begin{figure}[h!]
\captionsetup[subfigure]{position=b}
\centering
 \begin{minipage}[b]{.45\linewidth}
    \includegraphics[width=1\textwidth]{figures/ci/dataMc/figaux_05a.pdf}
    \subcaption{}\label{fig:1a}
\end{minipage}
\begin{minipage}[b]{.45\linewidth}
    \includegraphics[width=1\textwidth]{figures/ci/dataMc/figaux_05b.pdf}
    \subcaption{}
\end{minipage} \\
\begin{minipage}[b]{.45\linewidth}
    \includegraphics[width=1\textwidth]{figures/ci/dataMc/figaux_06a.pdf}
    \subcaption{}
\end{minipage}
\begin{minipage}[b]{.45\linewidth}
    \includegraphics[width=1\textwidth]{figures/ci/dataMc/figaux_06b.pdf}
    \subcaption{}
\end{minipage}
\caption{Invariant-mass distributions in the $ee$ channel (top) and $\mu\mu$ channel (bottom). Plots on the left show selected constructive CI signal shapes imposed on top of the simulated distribution, while plots on the right show the same for destructive CI signal shapes.}
\label{fig:ciMassMcPlot}
\end{figure}
\clearpage
}

% \subsection{Kinematic Distributions}

% Several distributions of kinematic variables are provided in the following figures.
% These plots show the distribution of simulated events along with data events for both \ee and \mm selections.
% The following figures show some kinematic variables for each selection.
% Fully simulated resonant signals are included in these as illustrations, as the CI signal is reweighted only in the invariant-mass distribution.


% \afterpage{
% \begin{figure}[h!]
% \captionsetup[subfigure]{position=b}
% \centering
%  \begin{minipage}[b]{.45\linewidth}
%     \includegraphics[width=1\textwidth]{figures/ci/dataMc/stacks_mc16e_2015-2018_ee_met_log100.pdf}
%     \subcaption{}\label{fig:1a}
% \end{minipage}
% \begin{minipage}[b]{.45\linewidth}
%     \includegraphics[width=1\textwidth]{figures/ci/dataMc/stacks_mc16e_2015-2018_ee_ptll_log100.pdf}
%     \subcaption{}
% \end{minipage} \\
% \begin{minipage}[b]{.45\linewidth}
%     \includegraphics[width=1\textwidth]{figures/ci/dataMc/stacks_mc16e_2015-2018_ee_phi1.pdf}
%     \subcaption{}
% \end{minipage}
% \begin{minipage}[b]{.45\linewidth}
%     \includegraphics[width=1\textwidth]{figures/ci/dataMc/stacks_mc16e_2015-2018_ee_phi2.pdf}
%     \subcaption{}
% \end{minipage}
% \caption{Kinematic distributions in the $ee$ channel. (a) $E_T^\text{miss}$, (b) dielectron \pt, leading electron $\phi$, and subleading electron $\phi$.}
% \label{fig:}
% \end{figure}
% \clearpage
% }

% \afterpage{
% \begin{figure}[h!]
% \captionsetup[subfigure]{position=b}
% \centering
% \begin{minipage}[b]{.45\linewidth}
%     \includegraphics[width=1\textwidth]{figures/ci/dataMc/stacks_mc16e_2015-2018_ee_eta1.pdf}
%     \subcaption{}
% \end{minipage} 
% \begin{minipage}[b]{.45\linewidth}
%     \includegraphics[width=1\textwidth]{figures/ci/dataMc/stacks_mc16e_2015-2018_ee_eta2.pdf}
%     \subcaption{}
% \end{minipage}\\
% \begin{minipage}[b]{.45\linewidth}
%     \includegraphics[width=1\textwidth]{figures/ci/dataMc/stacks_mc16e_2015-2018_ee_pt1_log100.pdf}
%     \subcaption{}
% \end{minipage}
% \begin{minipage}[b]{.45\linewidth}
%     \includegraphics[width=1\textwidth]{figures/ci/dataMc/stacks_mc16e_2015-2018_ee_pt2_log100.pdf}
%     \subcaption{}
% \end{minipage}
% \caption{Kinematic distributions in the $ee$ channel. (a) leading electron $\eta$, (b) subleading electron $\eta$, leading electron \pt, and subleading electron \pt.}
% \label{fig:}
% \end{figure}
% \clearpage
% }

% \afterpage{
% \begin{figure}[h!]
% \captionsetup[subfigure]{position=b}
% \centering
%  \begin{minipage}[b]{.45\linewidth}
%     \includegraphics[width=1\textwidth]{figures/ci/dataMc/stacks_mc16e_2015-2018_uu_met_log100.pdf}
%     \subcaption{}\label{fig:1a}
% \end{minipage}
% \begin{minipage}[b]{.45\linewidth}
%     \includegraphics[width=1\textwidth]{figures/ci/dataMc/stacks_mc16e_2015-2018_uu_ptll_log100.pdf}
%     \subcaption{}
% \end{minipage} \\
% \begin{minipage}[b]{.45\linewidth}
%     \includegraphics[width=1\textwidth]{figures/ci/dataMc/stacks_mc16e_2015-2018_uu_phi1.pdf}
%     \subcaption{}
% \end{minipage}
% \begin{minipage}[b]{.45\linewidth}
%     \includegraphics[width=1\textwidth]{figures/ci/dataMc/stacks_mc16e_2015-2018_uu_phi2.pdf}
%     \subcaption{}
% \end{minipage}
% \caption{Kinematic distributions in the $\mu\mu$ channel. (a) $E_T^\text{miss}$, (b) dielectron \pt, leading muon $\phi$, and subleading muon $\phi$.}
% \label{fig:}
% \end{figure}
% \clearpage
% }

% \afterpage{
% \begin{figure}[h!]
% \captionsetup[subfigure]{position=b}
% \centering
% \begin{minipage}[b]{.45\linewidth}
%     \includegraphics[width=1\textwidth]{figures/ci/dataMc/stacks_mc16e_2015-2018_uu_eta1.pdf}
%     \subcaption{}
% \end{minipage} 
% \begin{minipage}[b]{.45\linewidth}
%     \includegraphics[width=1\textwidth]{figures/ci/dataMc/stacks_mc16e_2015-2018_uu_eta2.pdf}
%     \subcaption{}
% \end{minipage}\\
% \begin{minipage}[b]{.45\linewidth}
%     \includegraphics[width=1\textwidth]{figures/ci/dataMc/stacks_mc16e_2015-2018_uu_pt1_log100.pdf}
%     \subcaption{}
% \end{minipage}
% \begin{minipage}[b]{.45\linewidth}
%     \includegraphics[width=1\textwidth]{figures/ci/dataMc/stacks_mc16e_2015-2018_uu_pt2_log100.pdf}
%     \subcaption{}
% \end{minipage}
% \caption{Kinematic distributions in the $\mu\mu$ channel. (a) leading muon $\eta$, (b) subleading muon $\eta$, leading muon \pt, and subleading electron \pt.}
% \label{fig:}
% \end{figure}
% \clearpage
% }

% \clearpage

\input{sections/contactInteractions-background}
\input{sections/contactInteractions-signal}
\input{sections/contactInteractions-syst}
\section{Statistical Analysis}\label{sec:ciStat}

Statistical methods are used to distill two types of information from the collected dataset.
First, to determine the probability that the observed data is incompatible with the background-only (B-only) hypothesis.
Second, to determine the smallest putative signal such that, if extant, would produce a signal+background (S+B) hypothesis that is incompatible with the observed data.
The former is answered by a significance test, described in Section \ref{sec:ciSigTest}, while the latter is answered by setting a limit, described in Section \ref{sec:ciLimitSetting}.

\subsection{Likelihood Ratio and CLs Method}

\begin{figure}[h!]
\captionsetup[subfigure]{position=b}
\centering
\subfloat[][]{\label{fig:ciClsNobs}{\includegraphics[width=0.5\textwidth]{figures/stats/stat-nobs.pdf}}}
\subfloat[][]{\label{fig:ciClsProfileLikelihood}{\includegraphics[width=0.5\textwidth]{figures/stats/stat-likel.pdf}}}
\caption{PDFs predicted by S+B and B-only hypotheses of test statistics (a) \nobs and (b) the likelihood ratio. Note the log scale of (b). In each case, the shaded regions mark the set of test statistic values for which each hypothesis would be more incompatible then with the observed value shown in grey.}
\label{fig:ciCls}
\end{figure}

% \begin{figure}[h!]
% \captionsetup[subfigure]{position=b}
% \centering
% \includegraphics[width=0.5\textwidth]{figures/ci/profileLikelihoodPlots/toys-n29-fc-lambda-ll-const-LL-nToy5000-nSteps50-seed8.png}
% \caption{}
% \label{fig:ciClsProfileLikelihood}
% \end{figure}

% Test statistic: likelihood ratio
The fundamental tool used to compare two hypotheses is the \emph{test statistic}, a quantity calculated from the observed data.
Each hypothesis predicts a probability density function (PDF) that describes the probability to observe values of the test statistic.
The PDF of the B-only hypothesis, $\mathcal{L}(\nobs)$, is a function of the test statistic \nobs.
Composite hypotheses are more useful and are defined by additional parameters, $\theta$, that may be estimated from the observation.
In general, a B-only hypothesis defined by $\theta_0$ predicts a PDF of $\mathcal{L}(\nobs|\theta_0)$, while an S+B hypothesis defined by $\theta_1$ predicts a PDF of $\mathcal{L}(\nobs|\theta_1)$.
For example, a B-only hypothesis may predict a Gaussian PDF with a mean that is smaller than the average predicted by an S+B hypothesis.
An illustration is shown in Figure \ref{fig:ciClsNobs}.
Here, the B-only hypothesis is more compatible with an observation of \nobs=500 than \nobs=1,000, while the converse is true of the S+B hypothesis.
The choice to accept or reject a hypothesis is made by a comparison of their respective PDFs.

While the test statistic may be any quantity calculated from data, an optimal choice for the test statistic may be made to resolve the difference between the two hypotheses.
A common choice is to use the ratio of the S+B and B-only hypotheses to define the \emph{likelihood ratio} in Equation \ref{eqn:ciLikelihoodTestStat}.
\begin{equation}\begin{split}\label{eqn:ciLikelihoodTestStat}
    \Lambda(\nobs)=\frac{\mathcal{L}(\nobs|\theta_1)}{\mathcal{L}(\nobs|\theta_0)},
\end{split}\end{equation} 
The Neyman-Person lemma states that the likelihood ratio test is the most likely to reject the B-only hypothesis, given that the S+B hypothesis is true \cite{eilam}

An example of the PDFs of the likelihood distributions, produced under the assumption of either the S+B or B-only hypotheses, is shown in figure \ref{fig:ciClsProfileLikelihood}.
Measurements of the likelihood ratio test statistic, $\Lambda(\nobs)$, can fall at different points on the horizontal axis.
As in the case of the earlier illustration, the two compatibility of the two hypothesis with the observation may then be assessed.
Data measured at larger values of $\Lambda(\nobs)$ are \emph{less compatible} with the background-only hypothesis.
Likelihood ratios can complicated functions; in practice they usually need be estimated computationally. 

% In practice, the quantity $-\ln{\Lambda(\nobs)}$ is used for computational simplicity.

% \begin{figure}[h!]
% \captionsetup[subfigure]{position=b}
% \centering
% \includegraphics[width=0.5\textwidth]{figures/ci/cls.png}
% \caption{}
% \label{fig:ciClsNobs}
% \end{figure}

% CLs
% The PDF of $\Lambda(\nobs)$ is defined under both the B-only and S+B hypotheses.
The \emph{CLs method} is a statistical convention used in particle physics to compare hypotheses.
Taking first the PDF under the B-only hypothesis, $\Lambda(\nobs|\theta_0)$.
The integral of the test statistic $\Lambda(\nobs|\theta_0)$ above a given observed value of $\nobs$ defines the \emph{p-value}, $p_0$, of the observation.
This is the probability, according to the B-only hypothesis, to observe a value of the test statistic that is less likely than the actual observed value.
The p-value is illustrated in the shaded blue areas of the plots in Figure \ref{fig:ciCls}.
The complement of the p-value, shown as the unshaded region under the blue curve, defines the value $\clb\equiv1-p_0$.
An analogous value, $\clsb$, is defined for the likelihood ratio under the S+B hypothesis, using the PDF $\Lambda(\nobs|\theta_1)$.
The value $p_1$ is the defined as the integral of $\Lambda(\nobs|\theta_1)$ above the observed value, and $\clsb\equiv1-p_1$.
Finally, the ratio of these two values defines the arbitrarily named value $\cls\equiv \clsb/\clb$.
This ratio is interpreted as the confidence in the S+B hypothesis compared to the B-only hypothesis \cite{read}.

\subsection{Statistical Model}\label{sec:ciStatModel}

Each statistical question is answered through the comparison of B-only and S+B hypotheses.
Three related tests are performed;
\begin{enumerate}
    \item the background-only hypothesis versus a generic model-independent hypothesis of signal events;
    \item the background-only hypothesis versus a contact interaction hypothesis involving either lepton channel;
    \item the background-only hypothesis versus a contact interaction hypothesis involving both lepton channels.
\end{enumerate}
Each of these hypotheses is described by one of the following likelihood functions.
The likelihood converts the hypothesis into an expression of the probability to observe a yield in an SR.
In each case, a \emph{parameter of interest} (POI) is used to define the signal hypothesis.
For the model-independent hypothesis of a generic signal production, the POI is the number of signal events produced in the SR, $N_s$.
For the contact interaction model, the POI is the energy scale \lam. Different values of \lam correspond to different S+B hypotheses.
Figure \ref{fig:ciNSigInSr} shows that, in the case of CI models, the number of signal events produced is a function of \lam: $N_s(\lam)$.
Comparisons of the model-independent results with the CI results provide a useful cross-check.

The first likelihoods describe the background-only hypothesis and a hypothesis predicting some number of signal events, $N_s$, to be reconstructed in the SR.
In this model, the POI is $N_s$.
The PDFs of the number of events to observe in the SR for each hypothesis are given in Equations \ref{eqn:ciNullLikelihoodNSig} and \ref{eqn:ciAltLikelihoodNSig}.
\begin{flalign}
\text{PDF}_\text{b}(\vec{\theta}) =& \text{Pois}((1+\theta_\text{b})\times N_b) \times \text{Gaus}(\theta_\text{b},\sigma_\text{b}) \label{eqn:ciNullLikelihoodNSig}\\
\text{PDF}_\text{s+b}(\vec{\theta}) =& \text{Pois}(N_s+(1+\theta_\text{b})\times N_b) \times \text{Gaus}(\theta_\text{b},\sigma_\text{b}) \label{eqn:ciAltLikelihoodNSig}
\end{flalign}
The functions $\text{Pois}(N_\text{exp})$ are Poisson probability distributions with medians $N_\text{exp}$.
In this case, $N_\text{exp}=(1+\theta_\text{b})\times N_b$, where $N_b$ is the expected background in the SR from the extrapolation procedure.
The parameter $\theta_\text{b}$ is a nuisance parameter that is fit to the data and corresponds to the measured uncertainty on $N_b$.
The functions $\text{Gaus}(\theta_\text{b},\sigma_\text{b})$ are Gaussian constraints on the nuisance parameter $\theta_\text{b}$. These have means centered at $\theta_\text{b}=0$, and standard deviations $\sigma_\text{b}$.
For these PDFs, which are constructed to be agnostic as to the form of the signal model, the uncertainty $\sigma_\text{b}$ is the sum in quadrature of the extrapolation uncertainty and the ISS.

Next are the PDFs for hypotheses describing contact interactions, limited to individual lepton channels.
Equations \ref{eqn:ciNullLikelihood} and \ref{eqn:ciAltLikelihood} give the probability distributions for B-only and S+B hypotheses, respectively.
\begin{flalign}
\text{PDF}_\text{b}(\vec{\theta}) =& \text{Pois}((1+\theta_\text{b})\times N_b) \times \text{Gaus}(\theta_\text{b},\sigma_\text{b}) \label{eqn:ciNullLikelihood}\\
\text{PDF}_\text{s+b}(\vec{\theta}) =& \text{Pois}((1+\theta_\text{s})\times N_s(\Lambda)+(1+\theta_\text{b})\times N_b) \times \notag \\
                                          & \text{Gaus}(\theta_\text{b},\sigma_\text{b}) \times \text{Gaus}(\theta_\text{s},\sigma_\text{s}) \label{eqn:ciAltLikelihood}
\end{flalign}
The standard deviations of the Gaussian constraints correspond to the uncertainties described in Section \ref{sec:ciSyst}. 
$\sigma_\text{s}$ is the experimental uncertainty on the signal yield.
$\sigma_\text{b}$ is the total uncertainty on the background yield, which consists of the sum in quadrature of the extrapolation, ISS, and the function bias uncertainties.
Each of these numbers is given in Table \ref{tab:ciUncerts}.
The nuisance parameters are seen to modify the signal and background expectations in the Poisson function via $(1+\theta)$ terms.
In these models, the parameter of interest, \lam, is used to determine the number of signal events expected in the SR.
This is performed with a smooth interpolation between the generated CI shapes to provide $N_s(\lam)$.
For each of the four signals, a set of PDFs is constructed for each chirality combination, leading to 16 total models.

Last are the hypotheses dealing with CI models in both lepton channels.
These hypotheses predict signal production in both \ee and \mm channels.
For each constructive or destructive SRs, the observations in both \ee and \mm SRs are mutually independent.
Therefore the combined likelihood is the product of the individual likelihoods for each lepton channel corresponding to Equations \ref{eqn:ciNullLikelihood} and \ref{eqn:ciAltLikelihood}.
The observations in the constructive and destructive SRs of the same lepton channel are not mutually independent and therefore are not combined.
Consequently, for each interference pattern, the likelihoods of observations in the two leptonic SRs are combined to produce a total likelihood.
This is repeated for each chirality, resulting in eight pairs of hypotheses.

For each B-only and S+B ($\text{PDF}_\text{b}$ or $\text{PDF}_\text{s+b}$) PDF given here, a corresponding likelihood ($\mathcal{L}(\nobs|\theta_0)$ or $\mathcal{L}(\nobs|\theta_1)$) exists.
In this form, the likelihood expresses the probability of observing $\nobs$ events given some nuisance parameters $\theta$.

% Frequintist throw toys
It is helpful in computing \cls values to have PDF shapes, under both B-only and S+B hypotheses, for the likelihood ratio test statistic given in Equation \ref{eqn:ciLikelihoodTestStat}.
The PDF's shape is determined straightforwardly from the B-only and S+B PDF shapes with a frequentist Monte-Carlo procedure.
A number of pseudo-observations are generated from each hypothesis PDF ($\text{PDF}_\text{b}$ or $\text{PDF}_\text{s+b}$).
The nuisance parameters are allowed to vary corresponding to the width of their corresponding Gaussian constraint.
The number of observed events is sampled from the Poisson term.

The test statistic is calculated for each pseudo-observation.
This is simply a matter of fitting nuisance parameters of the appropriate likelihood ($\mathcal{L}(\nobs|\theta_0)$ or $\mathcal{L}(\nobs|\theta_1)$) to the pseudo-observation, and noting the probability of that observation.
This leads to distributions of the test statistic under each hypothesis, as shown in Figure \ref{fig:ciClsProfileLikelihood}.

\subsection{Significance test}\label{sec:ciSigTest}

\begin{figure}[h!]
\captionsetup[subfigure]{position=b}
\centering
\subfloat[][]{{\includegraphics[width=0.24\textwidth]{figures/ci/nEventsTestStat/ee-const.png}}}
\subfloat[][]{{\includegraphics[width=0.24\textwidth]{figures/ci/nEventsTestStat/ee-dest.png}}}
\subfloat[][]{{\includegraphics[width=0.24\textwidth]{figures/ci/nEventsTestStat/mm-const.png}}}
\subfloat[][]{{\includegraphics[width=0.24\textwidth]{figures/ci/nEventsTestStat/mm-dest.png}}}
\caption{Predictions of the B-only hypotheses estimated using ensembles of pseudo-experiments. The p-value is indicated in the shaded red region of the distribution. (a) and (b) show \ee constructive and destructive predictions, while (d) and (e) show \mm constructive and destructive predictions.}
\label{fig:ciSignificance}
\end{figure}

The significance of the data, given the background-only hypothesis, is evaluated by considering the p-value.
While it is possible to use the likelihood test statistic in Equation \ref{eqn:ciLikelihoodTestStat}, the signal dependence is undesirable.
Instead, the number of events yielded in the SR, \nobs, is used as the test statistic.
The corresponding p-value is the probability of observing a yield at least as large as that seen in data.
Figure \ref{fig:ciSignificance} shows the distributions of the event yields predicted by the background-only hypotheses in each SR.
The observed number of events is illustrated, and the integral corresponding to the p-value is highlighted.

The PDF distributions of \nobs is produced using a frequentist approach.
The shape is approximated with one hundred thousand pseudo-experiments drawn from the B-only hypothesis likelihood given in Equation \ref{eqn:ciNullLikelihood}.

Probabilities of observations are often cited in terms of standard deviations from the mean with respect to the normal distribution.
The background \emph{significance} of a p-value is defined as the inverse of the cumulative distribution function of the upper tail of the standard normal distribution.
This is illustrated for each SR in Figure \ref{fig:ciSignificance}.

\subsection{Limit test}\label{sec:ciLimitSetting}

Limit tests are a generalization of the significance test. %, where a set of signal+background hypotheses are rejected due to their incompatibility with the data.
In this context, the B-only hypothesis is taken to be the signal+background hypothesis, and the S+B hypothesis is defined as the background-only hypothesis $\text{PDF}_\text{b}$.
The hypothesis test then seeks to reject the signal+background hypothesis in favor of the background-only hypothesis due to their incompatibility with the observed data.
Values of the POI describe the set of signal+background hypotheses to be considered.
For the purpose of this analysis, this means the goal of the limit setting procedure is to find the limiting POI value that predicts the smallest non-rejected signal contribution to the SR.
This is done using a series of hypothesis tests scanned over a range of POI values. \footnote{An animated illustration of these scans may be found: \url{http://hg8i.com/thesis/likelihoods/}.}
This value is reported as the \emph{limit} on the POI.
Values of the POI that predict larger signal contributions to the SR describe excluded signal models, while values of the POI that predict fewer signal events in the SR remain admissible.

The compatibility of hypotheses with respect to the observation is measured using the \cls method.
The value of \cls plays a similar role as would a p-value.
For a particular value of the POI, if the \cls value exceeds 0.95, then the signal+background hypothesis is considered rejected, and the value of the POI is considered excluded.

There are two types of hyperparameters of the limit setting procedure.
First is the range and resolution of values to consider in a scan over the POI.
A broader range with finer resolution adds accuracy but also computational expense to the resulting limit.
This reaches a point of diminishing returns when the accuracy of the limit exceeds the second-order uncertainties on the systematics.
When the limiting value of the POI falls between steps in the POI scan, an interpolation is performed between the steps.
In general, 30 to 50 steps are sufficient for the results reported here.

The second type of hyperparameters defines the number of pseudo-observations, or toys, to calculate the PDF shapes for the test statistic.
The likelihood ratio (Equation \ref{eqn:ciLikelihoodTestStat}) serves as the test statistic for all limit setting.
The reliability of the results is quite sensitive to the number of toys.
Noting the log scale of Figure \ref{fig:ciClsProfileLikelihood} illustrates this point; many toys are needed to sample the tails of the likelihood distributions smoothly.
In this analysis, the limits were found to converge to a relative accuracy of $10^{-2}$ between one and two hundred thousand toys.
This 
For the results of this analysis, four hundred thousand toys were used for each POI step.

\input{sections/contactInteractions-results}
\section{Summary}\label{sec:ciConclusion}

A search for new physics in non-resonant dilepton production via the dielectron and dimuon invariant-mass spectra has been presented.
This search made use of the full 139 fb$^{-1}$ of proton--proton collision data collected by ATLAS during Run~2 of the LHC at $\sqrt{s}=13$~TeV.
No significant excess in the collected data is observed above the expected background.
Upper limits are set on the \xsbr of new signal processes and lower limits on the CI scale \lam.
The limits on \xsbr are easily reinterpreted in terms of new physics models.
This is the first time such results have been made available.
\footnote{Precise values for the limits are publicly accessible: \url{https://www.hepdata.net/record/ins1802523}.}
The limits on \lam are the most robust frequentist limits ever set on contact interaction models.

\begin{figure}[htb]
\centering
\includegraphics[width=0.70\textwidth]{figures/ci/results/hist-lambda-ll-Lambda.pdf}
\caption{
Comparison the new observed limits on \lam, shown in black, with similar \ll observations by ATLAS and other experiments.
The ATLAS results listed use datasets $\sqrt{s}=13$~TeV 36.1 fb$^{-1}$ \cite{EXOT-2016-05}, $\sqrt{s}=13$~TeV 3.1 fb$^{-1}$ \cite{EXOT-2015-07}, $\sqrt{s}=8$~TeV 20 fb$^{-1}$ \cite{EXOT-2013-19}, and $\sqrt{s}=7$~TeV 5.0 fb$^{-1}$ \cite{EXOT-2012-17}.
The most recent CMS result $\sqrt{s}=13$~TeV 36 fb$^{-1}$ is shown in red \cite{cmsCi}.
Several older studies set limits on combined LR+RL chirality models, which are not comparable to those set in this work.
The older ZEUS \cite{zeusCi} and ALEPH \cite{alephCi} results appear at the bottom.
}
\label{fig:ciHistoricalLimits}
\end{figure}

% Improvements
A number of new techniques were developed in order to enable the production of this result.
% Data driven
Most significantly, the results make use of a background estimate derived from the data in a low mass control region.
This approach replaces theoretical and experimental uncertainties with well studied statistical uncertainties on the background estimate.
These uncertainties are measured directly and robustly.
In particular, a new method for measuring spurious signal has been introduced with the ISS procedure.
% Frequentest
Additionally, the limits on both \xsbr and \lam are set using a frequentist approach.
This eliminates arbitrary prior probabilities on signal models.
These techniques, along with the integrated luminosity of the full Run~2 dataset, allow this search to probe unprecedented energy and length scales.
The strongest limits are set on the combined left-left chirality constructive model.
These observed (expected) limits exclude this model for \lam up to 35.8 (27.6)~TeV at 95\% CL.


The results of this analysis are placed in a broader context with other studies in Figure \ref{fig:ciHistoricalLimits}.
This plot shows the most recent observed limits along with earlier results by the ATLAS, CMS, ZEUS, and ALEPH collaborations from the past two decades.
The new limits on \lam use new techniques and statistical interpretation to replicate prior exclusions.
The excluded region, illustrated by the shaded grey region, has been expanded and pushes the limit on \lam higher than any previous comparable limit.







