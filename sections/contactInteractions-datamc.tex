\section{Dilepton Event Selection}\label{sec:ciEvSel}
The present search is concerned with collisions that produce pairs leptons.
This section lists selection criteria used to identify such events.
% from the dataset collected during Run~2 of the LHC.
The observed dataset, which consists of the events collected by ATLAS during  Run~2 of the LHC, is detailed along with the corresponding simulated background and signal datasets.
% Finally, comparisons between the recorded data and simulation are provided.


\subsection{Event Selection}
During Run~2, roughly $10^{16}$ proton collisions took place inside the ATLAS experiment.
The majority of these events are uninteresting for the purpose of this analysis, so only events meeting appropriate criteria are considered.
This reduces the total number of data events considered for the analysis to 754,870 dimuon events and 883,594 dielectron events.

% GRL
Only events recorded during good operation of the detector are used.
The events meeting this requirement comprise the Good Run List, summarized in Section \ref{sec:physData}.

% Trigger
The first requirement for an event to be considered is the trigger: only events identified as interesting by the ATLAS trigger system are recorded.
The triggers used during data collection differ from year to year. 
In the dielectron channel, the following trigger requirements are applied.
\begin{itemize}
	\item 2015: Two electrons with $\et>12$~GeV,
	\item 2016: Two electrons with $\et>17$~GeV,
	\item 2017 and 2018: Two electrons with $\et>24$~GeV.
	% \item 2015: 2e12\_lhloose\_L12EM10VH
	% \item 2016: 2e17\_lhvloose\_nod0
	% \item 2017 and 2018: 2e24\_lhvloose\_nod0
\end{itemize}
Although events passing these triggers are expected to contain two electrons, both may not be reconstructed after the event is fully processed. 
Therefore, subsequent criteria require at least two electrons to be reconstructed.

In the dimuon channel, the following trigger requirements are applied.
\begin{itemize}
	\item 2015: One isolated ($\ptconeThirty/\pt<0.06$) muon with $\pt>26$~GeV, or any non-isolated muon with $\pt>50$~GeV,
	\item 2016, 2017 and 2018: The same requirement, except the isolation uses \ptvarconeMuon.
	% \item 2015: mu26\_imedium or mu50
	% \item 2016, 2017 and 2018: mu26\_ivarmedium or mu50
\end{itemize}
These trigger on events with single isolated muons.
These triggers are used, rather than a muon equivalent to the electron triggers, to increase the trigger's efficiency for dimuon events; the requirement for an event to have two muons is enforced in the later.

% Object selection
After passing the trigger requirement, events are evaluated under selection criteria.
In events where multiple vertices are reconstructed, the vertex with the largest $\sum\pt^2$ defines the \emph{primary vertex}.
Events are required to have at least two Inner Detector tracks associated with the primary vertex.
The first step is to define requirements for which physical objects are to be considered in each event. This step follows the object definitions from Section \ref{sec:physObjects}.
% Many of the terms used here follow the definitions found in that section.

Further requirements are made as to where the objects were reconstructed in the detector. 
This defines the fiducial region in which the search is carried out.
This definition differs for electrons and muons.

Electrons are defined using the \code{Medium} likelihood identification.
They are required to pass \code{Gradient} isolation.
Additionally, they must not be from a dead calorimeter cluster.
An additional \emph{loose selection} for electrons is defined to study the background from objects falsely reconstructed as electrons.
For these electrons, the \code{LooseAndBLayer} LH identification replaces the \code{Medium} LH.
This is otherwise the same as the nominal electron selection.
The kinematic criteria for both electron selections are listed in Table \ref{tab:ciElectronSel}.

\begin{table}[!htb]
\caption{Selection criteria for electrons. Parameters $d_{0}$ and $z_{0}$ are the transverse and longitudinal displacements of the track associated with the electron, and the vertex.}
\begin{center}
    \begin{tabular}[ht]{l l}
    \toprule
    Feature & Criteria \\
    \midrule
    Pseudorapidity range & $(|\eta| < 1.37) \quad || \quad (1.52 < |\eta| < 2.47)$ \\
    Transverse momentum & p$_T$ $>$ 30~GeV \\
    Track impact parameter significance & ${|d_{0}^{BL}|\over\sigma}$ $<$ 5 \\
    Track $z$ displacement & $|\Delta z_{0}^{BL} \sin{\theta}| <$ 0.5~mm \\
    \bottomrule
    \end{tabular}
\end{center}
\label{tab:ciElectronSel}
\end{table}

Muons are defined using the \code{High}-$p_T$ selection working point and must pass the isolation requirement \code{FCTightTrackOnly}.
An additional cut, the bad muon veto, is used to reject muons with poorly measured \pt.
The remaining kinematic criteria for muons are given in Table \ref{tab:ciMuonsSel}.

\begin{table}[ht]
\caption{Selection criteria for muons. Parameters $d_{0}$ and $z_{0}$ are the transverse and longitudinal displacements of the track associated with the muon, and the vertex.}
\begin{center}
    \begin{tabular}[ht]{l l}
    \toprule
    Feature & Criteria \\
    \midrule
    Transverse momentum  & $\pt>30$ GeV\\
    Pseudorapidity range & $|\eta|<2.5$ \\
    Track impact parameter significance & ${|d_{0}^{BL}|\over\sigma}< 3$ \\
    Track $z$ displacement  & $|\Delta z_{0}^{BL} \sin{\theta}| < 0.5~mm$\\
    \bottomrule
    \end{tabular}
\end{center}
\label{tab:ciMuonsSel}
\end{table}

Occasionally, the interaction of a single particle with detectors results in the reconstruction of two particles.
To limit this occurrence, an \emph{overlap removal} scheme removes particles that are suspiciously close to each other.
The criteria are listed in Table \ref{tab:ciOr}.
\begin{table}[ht]
\caption{Overlap removal}
\begin{center}
    \begin{tabular}[ht]{l l l}
    \toprule
    Reject & Against & Criteria \\
    \midrule
    Electron & Electron & Shared ID track, $\pt^1<\pt^2$ \\
    Muon     & Electron & Is calo-muon and shared ID track \\
    Electron & Muon     & Shared ID track \\
    \bottomrule
    \end{tabular}
\end{center}
\label{tab:ciOr}
\end{table}
Further rejection of muons and electrons takes place if a jet is reconstructed within an angular distance $\Delta R<0.4$.
This helps reduce the presence of secondary leptons.

% Event selection
These criteria reduce the full set of recorded events to a subset to consider, and within each event a set of physical objects to analyze.
It remains to determine whether the event is interesting for the purpose of this dilepton analysis.
Only events containing either two electrons or two muons meet this threshold.
Of the same-flavor leptons in the event, the leading and subleading \et (\pt) pair are selected in the electron (muon) channel.
In the muon channel, only pairs of oppositely charged muons are considered. 
In the electron channel, the charge is ignored because bremsstrahlung emission of photons.
Such photons can alter the track of an electron, leading to the mis-identification of its charge.
Finally, a preliminary invariant mass cut of $\mll>130$~GeV is required.
In the case where both a dielectron and dimuon candidate meet these requirements, the dielectron is selected, and the dimuon is discarded.
This choice is made due to the superior resolution for high-\et electrons.

\subsection{Data and Simulation}
% The data yield rate, broken into the different runs and periods for each year, are shown in Figure ~\ref{fig:ciYields}.
% These plots count events after applying the full selections.

% \begin{figure}[ht!]
% \captionsetup[subfigure]{position=b}
% \centering
% \subfloat[][]{{\includegraphics[width=0.48\textwidth]{figures/ci/dataMc/compare_data_yields2015.pdf}}}
% \subfloat[][]{{\includegraphics[width=0.48\textwidth]{figures/ci/dataMc/compare_data_yields2016.pdf}}}\\
% \subfloat[][]{{\includegraphics[width=0.48\textwidth]{figures/ci/dataMc/compare_data_yields2017.pdf}}}
% \subfloat[][]{{\includegraphics[width=0.48\textwidth]{figures/ci/dataMc/compare_data_yields2018.pdf}}}
% \caption{Data yields for the each run period for the inclusive $ee$ (above) and $\mu\mu$ (below) selections.}
% \label{fig:ciYields}
% \end{figure}
% \clearpage

The data used in this analysis were collected during the LHC Run 2 from \sqrts=13~TeV proton-proton collisions.
The recorded integrated luminosity of the collisions is $139.0\pm2.4$~\fb \cite{ATLAS-CONF-2019-021}.

Despite the reliance on background estimates derived from data, this analysis uses simulated invariant-mass distributions for three purposes.
The first use is to model the CI signal. This is done using simulated DY events, reweighted to include interference and direct production from a contact interaction.
The second use is to test a variety of choices made during the analysis. In particular, the simulation informs the choice of a functional form that matches the expected background shape. Simulation is also used to optimally select the control and signal regions to maximize expected sensitivity while avoiding potential biases.
The third use is to measure the impact of experimental and theoretical uncertainties on the results.

All simulation-based background contributions are scaled by their respective cross-sections and summed to obtain the simulated background invariant-mass distribution.
The main backgrounds in decreasing order of contribution to the full mass spectrum are the Drell--Yan (DY) process, top-quark pair production ($t\bar{t}$), single-top-quark production, and diboson production.
The multi-jet and $W+$jets processes in the dielectron channel are estimated from the data using the matrix method \cite{EXOT-2016-05}. The contribution of such processes to the analysis is estimated using a likelihood fit, and is later treated as an uncertainty in the simulated background.
The same processes in the dimuon channel, as well as processes with $\tau$-leptons in both channels, have been measured to have a negligible impact and consequently are not considered.
The event generators for the hard-scattering process and the programs used for parton showering are listed in Table~\ref{tab:MC} with their respective parton distribution functions (PDFs).
Afterburner generators such as \textsc{Photos}~\cite{Golonka:2005pn} for the final-state photon radiation (FSR) modeling, \textsc{MadSpin}~\cite{Artoisenet:2012st} to preserve top-quark spin correlations, and \textsc{EvtGen}~\cite{Lange:2001uf} for the modeling of $c$- and $b$-hadron decays, are also included in the simulation.

\begin{table}[htbp]
\caption{The programs and PDFs used to generate the hard-scatter matrix element (ME) and to simulate parton showering (PS) in the signal and background processes.
\centering
The top-quark mass is set to 172.5 GeV.}
{\scriptsize
\begin{tabular}{lll}
\toprule
Background Process & ME Generator and ME PDFs & PS and non-perturbative effect with PDFs \\\hline
NLO Drell--Yan & \POWHEGBOX~, CT10~, \textsc{Photos} & \PYTHIAV{v8.186}~, CTEQ6L1~, \textsc{EvtGen1.2.0} \\
$t\bar{t}$  & \POWHEGBOX, NNPDF3.0NLO~ & \PYTHIAV{v8.230}, NNPDF23LO~, \textsc{EvtGen1.6.0} \\
Single top $s$-channel, $Wt$& \POWHEGBOX, NNPDF3.0NLO & \PYTHIAV{v8.230}, NNPDF23LO, \textsc{EvtGen1.6.0} \\
Single top $t$-channel & \POWHEGBOX, NNPDF3.04fNLO, \textsc{MadSpin} & \PYTHIAV{v8.230}, NNPDF23LO, \textsc{EvtGen1.6.0}  \\
Diboson ($WW$, $WZ$ and $ZZ$) & \SHERPA 2.1.1~, CT10 &\SHERPA 2.1.1, CT10  \\\hline
Signal Process & & \\\hline
LO Drell--Yan & \PYTHIAV{v8.186}, NNPDF23LO  &  \PYTHIAV{v8.186}, NNPDF23LO, \textsc{EvtGen1.2.0} \\
LO CI & \PYTHIAV{v8.186}, NNPDF23LO  &  \PYTHIAV{v8.186}, NNPDF23LO, \textsc{EvtGen1.2.0} \\
\bottomrule
\end{tabular}
}
\normalsize
\label{tab:MC}
\end{table}


The DY~\cite{ATL-PHYS-PUB-2016-003} and diboson~\cite{ATL-PHYS-PUB-2016-002} samples were generated in slices of dilepton mass to increase the sample statistics in the high-mass region.
Next-to-next-to-leading-order (NNLO) corrections in QCD and next-to-leading-order (NLO) corrections in EW were calculated and applied to the DY events.
The corrections were computed with {\textsc{VRAP}} v0.9~\cite{vrap} and the CT14 NNLO PDF set~\cite{CT14} in the case of QCD effects, whereas they were computed with {\textsc{MCSANC}}~\cite{MCSANC} in the case of quantum electrodynamic effects due to initial-state radiation, interference between initial- and final-state radiation, and Sudakov logarithm single-loop corrections.
These are calculated as mass-dependent scale factors that are applied to simulated events before reconstruction.
The top-quark samples~\cite{ATL-PHYS-PUB-2016-020} are normalized to the cross-sections calculated at NNLO in QCD, including resummation of the next-to-next-to-leading logarithmic soft gluon terms using \textsc{Top++}2.0 \cite{Czakon:2011xx}.

All fully simulated event samples include the effect of multiple proton interactions in the same or neighboring bunch crossings.
These effects are collectively referred to as pile-up.
The simulation of pile-up collisions was performed with \PYTHIAV{v8.186} using the ATLAS A3 set of tuned parameters~\cite{ATL-PHYS-PUB-2016-017} and the NNPDF23LO PDF set, and weighted to reproduce the average number of pile-up interactions per bunch crossing observed in data.
The generated events were passed through a full detector simulation~\cite{SOFT-2010-01} based on\ \GEANT~4~\cite{geant}.

% Scale factors
The simulated data is weighted by several scale factors (SF) to improve its representation of the observed data.
Pile-up weights are used to describe the effects multiple collisions per beam crossing.
Mass dependant $K$-factors account for differences in the total cross-section if higher-order calculations are available for a given process compared to the order available for simulation. In the case of the LO and NLO DY samples, the SFs provide corrections for higher-order QCD, EW, and photon-induced (PI) effects.
Experimental scale factors for leptons are considered. For electrons, reconstruction, trigger, isolation, and identification scales factors are applied. For muons, reconstruction, trigger, isolation, and track-to-vertex association (TTVA) scale factors are applied.
Trigger scale factors according to the specific channel.

% \begin{itemize}
%     \item Pile-up weights describe the effects multiple collisions per beam crossing.
%     \item Mass dependant $K$-factors account for differences in the total cross-section if higher-order calculations are available for a given process compared to the order available for simulation. In the case of the LO and NLO DY samples, the SFs provide corrections for higher-order QCD, EW, and photon-induced (PI) effects.
%     \item Experimental scale factors for leptons are considered. For electrons, reconstruction, trigger, isolation, and identification scales factors are applied. For muons, reconstruction, trigger, isolation, and track-to-vertex association (TTVA) scale factors are applied.
%     \item Trigger scale factors according to the specific channel.
% \end{itemize}

To reduce statistical uncertainties, a large additional DY sample is used where the detector response is modeled by smoothing the dilepton invariant-mass with mass-dependent acceptance and efficiency corrections, instead of using the computationally expensive \GEANT~4 simulation.
The relative dilepton mass resolution used in the smearing procedure is defined as $(m_{\ell\ell}-m_{\ell\ell}^\mathrm{true})/m_{\ell\ell}^\mathrm{true}$, where $m_{\ell\ell}^\mathrm{true}$ is the generated dilepton mass at Born level before final-state radiation.
The mass resolution is parameterized as a sum of a Gaussian distribution, which describes the detector response, and a Crystal Ball function composed of a secondary Gaussian distribution with a power-law low-mass tail, which accounts for bremsstrahlung effects and for the effect of poor resolution in the muon momentum at high-\pt.
The parameterization of the relative dilepton mass resolution as a function of $m_{\ell\ell}^\mathrm{true}$ is determined by a fit of the function described above to simulated DY events at NLO.
A similar procedure is used to produce a mass-smeared $t\bar{t}$ sample.
These two samples replace the equivalent ones produced with the full detector simulation wherever applicable in the remainder of the analysis.
These samples are composed of over 55 times the number of events in the observed dataset.

Signal \mll distribution shapes are obtained by a matrix element reweighting of the leading-order (LO) DY samples generated in slices of dilepton mass \cite{EXOT-2016-05}.
This reweighting includes the full interference between the non-resonant signal and the background DY process.
The weight function is the ratio of the analytical matrix elements of the full contact interaction (including the DY component) and the DY process only, both at LO precision.
It takes as an input the generated dilepton mass at Born level before FSR, the flavor of the incoming quarks, and the CI model parameters (\lam, chirality states, and the sign of interference).
These weights are applied to the LO DY events to transform these into the CI signal shapes, in steps of $2$~TeV between $\Lambda=12$~TeV and $\Lambda=100$~TeV.
Mass-dependent higher-order QCD production corrections for the signals are computed with the same methodology as for the DY background, correcting from LO to NNLO precision.
Similarly, electroweak corrections for the signals are applied in the CI reweighting along with the interference effects, correcting from LO to NLO precision.
These signal shapes are used for optimizations as well as for calculations of the cross-section and acceptance times efficiency.

The invariant-mass distributions of the simulated datasets, and of the observed data are shown in Figure \ref{fig:ciMassMcPlot}.
Several representative contact interaction shapes imposed on top of the background yields for reference.
These plots clearly show the relative composition of the background in the simulated distributions.
The plots of the \ee selection additionally include the multi-jet and $W+$jets background.

\afterpage{
\begin{figure}[h!]
\captionsetup[subfigure]{position=b}
\centering
 \begin{minipage}[b]{.45\linewidth}
    \includegraphics[width=1\textwidth]{figures/ci/dataMc/figaux_05a.pdf}
    \subcaption{}\label{fig:1a}
\end{minipage}
\begin{minipage}[b]{.45\linewidth}
    \includegraphics[width=1\textwidth]{figures/ci/dataMc/figaux_05b.pdf}
    \subcaption{}
\end{minipage} \\
\begin{minipage}[b]{.45\linewidth}
    \includegraphics[width=1\textwidth]{figures/ci/dataMc/figaux_06a.pdf}
    \subcaption{}
\end{minipage}
\begin{minipage}[b]{.45\linewidth}
    \includegraphics[width=1\textwidth]{figures/ci/dataMc/figaux_06b.pdf}
    \subcaption{}
\end{minipage}
\caption{Invariant-mass distributions in the $ee$ channel (top) and $\mu\mu$ channel (bottom). Plots on the left show selected constructive CI signal shapes imposed on top of the simulated distribution, while plots on the right show the same for destructive CI signal shapes.}
\label{fig:ciMassMcPlot}
\end{figure}
\clearpage
}

% \subsection{Kinematic Distributions}

% Several distributions of kinematic variables are provided in the following figures.
% These plots show the distribution of simulated events along with data events for both \ee and \mm selections.
% The following figures show some kinematic variables for each selection.
% Fully simulated resonant signals are included in these as illustrations, as the CI signal is reweighted only in the invariant-mass distribution.


% \afterpage{
% \begin{figure}[h!]
% \captionsetup[subfigure]{position=b}
% \centering
%  \begin{minipage}[b]{.45\linewidth}
%     \includegraphics[width=1\textwidth]{figures/ci/dataMc/stacks_mc16e_2015-2018_ee_met_log100.pdf}
%     \subcaption{}\label{fig:1a}
% \end{minipage}
% \begin{minipage}[b]{.45\linewidth}
%     \includegraphics[width=1\textwidth]{figures/ci/dataMc/stacks_mc16e_2015-2018_ee_ptll_log100.pdf}
%     \subcaption{}
% \end{minipage} \\
% \begin{minipage}[b]{.45\linewidth}
%     \includegraphics[width=1\textwidth]{figures/ci/dataMc/stacks_mc16e_2015-2018_ee_phi1.pdf}
%     \subcaption{}
% \end{minipage}
% \begin{minipage}[b]{.45\linewidth}
%     \includegraphics[width=1\textwidth]{figures/ci/dataMc/stacks_mc16e_2015-2018_ee_phi2.pdf}
%     \subcaption{}
% \end{minipage}
% \caption{Kinematic distributions in the $ee$ channel. (a) $E_T^\text{miss}$, (b) dielectron \pt, leading electron $\phi$, and subleading electron $\phi$.}
% \label{fig:}
% \end{figure}
% \clearpage
% }

% \afterpage{
% \begin{figure}[h!]
% \captionsetup[subfigure]{position=b}
% \centering
% \begin{minipage}[b]{.45\linewidth}
%     \includegraphics[width=1\textwidth]{figures/ci/dataMc/stacks_mc16e_2015-2018_ee_eta1.pdf}
%     \subcaption{}
% \end{minipage} 
% \begin{minipage}[b]{.45\linewidth}
%     \includegraphics[width=1\textwidth]{figures/ci/dataMc/stacks_mc16e_2015-2018_ee_eta2.pdf}
%     \subcaption{}
% \end{minipage}\\
% \begin{minipage}[b]{.45\linewidth}
%     \includegraphics[width=1\textwidth]{figures/ci/dataMc/stacks_mc16e_2015-2018_ee_pt1_log100.pdf}
%     \subcaption{}
% \end{minipage}
% \begin{minipage}[b]{.45\linewidth}
%     \includegraphics[width=1\textwidth]{figures/ci/dataMc/stacks_mc16e_2015-2018_ee_pt2_log100.pdf}
%     \subcaption{}
% \end{minipage}
% \caption{Kinematic distributions in the $ee$ channel. (a) leading electron $\eta$, (b) subleading electron $\eta$, leading electron \pt, and subleading electron \pt.}
% \label{fig:}
% \end{figure}
% \clearpage
% }

% \afterpage{
% \begin{figure}[h!]
% \captionsetup[subfigure]{position=b}
% \centering
%  \begin{minipage}[b]{.45\linewidth}
%     \includegraphics[width=1\textwidth]{figures/ci/dataMc/stacks_mc16e_2015-2018_uu_met_log100.pdf}
%     \subcaption{}\label{fig:1a}
% \end{minipage}
% \begin{minipage}[b]{.45\linewidth}
%     \includegraphics[width=1\textwidth]{figures/ci/dataMc/stacks_mc16e_2015-2018_uu_ptll_log100.pdf}
%     \subcaption{}
% \end{minipage} \\
% \begin{minipage}[b]{.45\linewidth}
%     \includegraphics[width=1\textwidth]{figures/ci/dataMc/stacks_mc16e_2015-2018_uu_phi1.pdf}
%     \subcaption{}
% \end{minipage}
% \begin{minipage}[b]{.45\linewidth}
%     \includegraphics[width=1\textwidth]{figures/ci/dataMc/stacks_mc16e_2015-2018_uu_phi2.pdf}
%     \subcaption{}
% \end{minipage}
% \caption{Kinematic distributions in the $\mu\mu$ channel. (a) $E_T^\text{miss}$, (b) dielectron \pt, leading muon $\phi$, and subleading muon $\phi$.}
% \label{fig:}
% \end{figure}
% \clearpage
% }

% \afterpage{
% \begin{figure}[h!]
% \captionsetup[subfigure]{position=b}
% \centering
% \begin{minipage}[b]{.45\linewidth}
%     \includegraphics[width=1\textwidth]{figures/ci/dataMc/stacks_mc16e_2015-2018_uu_eta1.pdf}
%     \subcaption{}
% \end{minipage} 
% \begin{minipage}[b]{.45\linewidth}
%     \includegraphics[width=1\textwidth]{figures/ci/dataMc/stacks_mc16e_2015-2018_uu_eta2.pdf}
%     \subcaption{}
% \end{minipage}\\
% \begin{minipage}[b]{.45\linewidth}
%     \includegraphics[width=1\textwidth]{figures/ci/dataMc/stacks_mc16e_2015-2018_uu_pt1_log100.pdf}
%     \subcaption{}
% \end{minipage}
% \begin{minipage}[b]{.45\linewidth}
%     \includegraphics[width=1\textwidth]{figures/ci/dataMc/stacks_mc16e_2015-2018_uu_pt2_log100.pdf}
%     \subcaption{}
% \end{minipage}
% \caption{Kinematic distributions in the $\mu\mu$ channel. (a) leading muon $\eta$, (b) subleading muon $\eta$, leading muon \pt, and subleading electron \pt.}
% \label{fig:}
% \end{figure}
% \clearpage
% }

% \clearpage
